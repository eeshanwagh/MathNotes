\ProvidesPackage{wmr}
%%%%
% Lecture Latex File for Mathematics Lectures
% By Eeshan Wagh
% Provides page geometry and headings, typographic style,
% and theorem environments.
%%%%

%%%%
% Package dependencies
% 
% * mathtools, amssymb, amsthm --- Standard math packages, we are setting math.



\RequirePackage{mathtools}
\RequirePackage{amssymb}
\RequirePackage{amsthm}
\RequirePackage{amsfonts}
% * cite --- better citations
\RequirePackage{cite}
% * url --- URL support
\RequirePackage{url}
% * sectsty --- Allows us to set section styles within a documentclass, 
%   since writing our own .cls would clash with the combine package
\RequirePackage{sectsty}
% * abstract --- Use this package to control spacing and delimeters 
%   of the abstracts.
\RequirePackage[runin]{abstract}
% * geometry --- Default page geometry is for 2 pages, 2:3 margin ratios
\RequirePackage{geometry}
% * fancyhdr --- Set headers
\RequirePackage{fancyhdr}
% * microtype --- microtypography package sets optical kerning, 
%   does better jobs of adjusting letter spacing on short lines
\RequirePackage[factor=1000]{microtype}
%%%%

%%%%
% Page geometry
\geometry{hmarginratio=2:3,nofoot=true,vmarginratio=2:3,inner=.1\paperwidth,top=.1\paperheight}
%%%%

%%%%
% Page headers
\pagestyle{fancy}
\fancyhead{}
\fancyhead[CO]{\textsc{The Waterloo Mathematics Review}}
\fancyhead[LO,RE]{\thepage}
\fancyfoot{}
%%%%

%%%%
% Bibliography style
\bibliographystyle{amsalpha}
%%%%

%%%%
% Section style and spacing
\sectionfont{\Large\mdseries\scshape\centering}
\subsectionfont{\normalfont\large\scshape\raggedright}
\subsubsectionfont{\normalfont\normalsize\scshape\raggedright}
\paragraphfont{\normalfont\normalsize\scshape}
\setlength{\parindent}{2em}
%%%%

%%%%
% Abstract style and spacing
\renewcommand{\abstractnamefont}{\normalfont\small\mdseries\scshape\flushleft}
\renewcommand{\abstracttextfont}{\normalfont\small}
\setlength{\absparindent}{0pt}
\setlength{\abstitleskip}{0pt}
\abslabeldelim{:\,}
%%%%

%%%%
% WMR Theorem and Remark style
\newtheoremstyle{wmrtheorem}
    {\topsep}
    {\topsep}
    {\normalfont \leftskip 2em }
    {-2em}
    {\mdseries\itshape}
    {.}
    { }
    {\thmname{#1}\thmnumber{ \mdseries\itshape#2}\thmnote{ \mdseries\itshape(#3)}}

\newtheoremstyle{wmrremark}
    {\topsep}
    {\topsep}
    {\normalfont}
    {}
    {\itshape}
    {:}
    { }
    {\thmname{#1}\thmnumber{ \itshape#2}\thmnote{ \itshape(#3)}}
%%%%

%%%%
% Theorem environments and numbering
\theoremstyle{wmrtheorem}
\newtheorem{theorem}{Theorem}[section]
\newtheorem*{auxiliar}{Theorem}
\newtheorem{corollary}[theorem]{Corollary}
\newtheorem{proposition}[theorem]{Proposition}

\newtheorem{lemma}{Lemma}[section]
\newtheorem{definition}{Definition}[section]
\newtheorem{example}{Example}[section]
\newtheorem{claim}{Claim}[section]
\newtheorem{openquestion}{Open Question}[section]
\newtheorem*{claim*}{Claim}

\newtheorem{axiom}{Axiom}

\theoremstyle{wmrremark}
\newtheorem{remark}[theorem]{Remark}

\numberwithin{equation}{section}
\numberwithin{figure}{section}
\numberwithin{table}{section}
%%%%

%%%%
% WMR Enumerate and Itemize, replace enumerate and itemize
% with these.
\newenvironment{wmrenumerate}{
\begin{list}{\arabic{enumi}.}{\usecounter{enumi}\setlength{\leftmargin}{\parindent}} }{\end{list}}
\newenvironment{wmritemize}{\begin{list}{\labelitemi}{\setlength{\leftmargin}{\parindent}}}{\end{list}}